\documentclass[a4paper]{article}
\usepackage[utf8]{inputenc}
\usepackage{polski}
\usepackage[polish]{babel}
\usepackage{graphicx}
\usepackage[labelsep=period]{caption}
\usepackage{subcaption}
\usepackage[export]{adjustbox}
\usepackage[margin=1in]{geometry}
\usepackage{float}
\usepackage{hyperref}
\usepackage[nochapter, owncaptions]{vhistory}

\renewcommand\thesection{\arabic{section}.}
\renewcommand\thesubsection{\arabic{section}.\arabic{subsection}.}
\renewcommand\thesubsubsection{\arabic{section}.\arabic{subsection}.\arabic{subsubsection}.}

\brokenpenalty=1000
\clubpenalty=1000
\widowpenalty=1000

\renewcommand{\vhversionname}{Wersja}
\renewcommand{\vhdatename}{Data}
\renewcommand{\vhauthorname}{Autor}
\renewcommand{\vhchangename}{Opis}

\author{Cezary Świtała}
\title{Specyfikacja funkcjonalna aplikacji e-sklep}

\begin{document}

\pagenumbering{gobble}
\makeatletter
\begin{titlepage}
    \centering

    \large Instytut Informatyki Uniwersytetu Wrocławskiego

    \vspace*{6cm}

    \large \@author

    \vspace{0.5cm}

    \huge \@title

    \vspace{0.5cm}

    \Large Kurs projektowania aplikacji z bazami danych 2021/2022

    \vspace{1.5cm}

    \vfill

    \large Wrocław, 25 kwietnia 2022
\end{titlepage}
\makeatother
\pagenumbering{arabic}

\setcounter{page}{2}

\noindent{\Large \textbf{Historia zmian}\par}

\begin{versionhistory}
  \vhEntry{1.0}{25.04.2022}{Cezary Świtała}{Utworzono dokument}
\end{versionhistory}

\newpage

\section{Wstęp}

Niniejszy dokument ma na celu określenie funkcjonalności aplikacji \emph{e-sklep} realizowanej w ramach przedmiotu \emph{Kurs projektowania aplikacji z bazami danych}, ze szczególnym uwzględnieniem wysokopoziomowego opisu głównego procesu biznesowego oraz omówienia wynikających z niego kluczowych funkcjonalności.

\section{Główny proces biznesowy}

Głównym procesem biznesowym aplikacji będzie proces dokonywania zakupu oferowanych przez sklep produktów.

\subsection{Proces zakupu z perspektywy klienta}

\begin{enumerate}
	\item Klient przegląda dostępne w sklepie produkty, wyświetla ich szczegółowy opis, zapoznaje się z opiniami innych użytkowników.
	\item Wybrane pozycje dodaje do koszyka.
	\item Po zakończeniu przeglądania zapoznaje się z listą produktów w koszyku, wybiera opcje dostawy oraz zobowiązuje się dokonać płatności.
	\item Klient przegląda listę złożonych zamówień. Może odczytać informacje o statusie ich realizacji.
	\item Klient otrzymuje zakupione produkty.
	\item Może podzielić się z innymi użytkownikami opinią dotyczącą usług i produktów.
\end{enumerate}

\subsection{Proces zakupu z perspektywy sprzedawcy}

\begin{enumerate}
	\item Sprzedawca dodaje, edytuje i usuwa produkty w ofercie sklepu.
	\item Przegląda zamówienia złożone w sklepie wraz z ich szczegółowym opisem.
	\item Może zmienić stan realizacji zamówienia.
	\item W stosownym momencie przekazuje produkty do wysłania.
	\item Sprzedawca może zapoznać się z opiniami związanymi z zamówieniami oraz poszczególnymi produktami.
\end{enumerate}

\section{Obiekty biznesowe}

Do głównych obiektów biznesowych zaliczają się:

\begin{itemize}
	\item \textbf{Sklep} -- działalność, która prowadzi sprzedaż internetową za pomocą 
	tworzonej aplikacji.
	\item \textbf{Produkt} -- przedmiot, którego sprzedaż oferuje sklep.
	\item \textbf{Użytkownik} -- każda osoba używająca aplikacji.
	\begin{itemize}
		\item \textbf{Klient} -- użytkownik uprawniony do przeglądania przedmiotów, 
		składania zamówień itp.
		\item \textbf{Sprzedawca} -- użytkownik uprawniony także do edytowania oferty 
		sklepu, stanu zamówień itp.
	\end{itemize}
	\item \textbf{Zamówienie} -- zbiór produktów związany z klientem, dostawą i 
	płatnością.
	\item \textbf{Koszyk} -- zbiór produktów, który użytkownik może edytować w trakcie 
	zakupów oraz na podstawie którego tworzy zamówienie.
	\item \textbf{Płatność} -- opis metody opłacenia zakupionych przedmiotów.
	\item \textbf{Dostawa} -- opis metody dostarczenia zakupionych produktów do klienta.
\end{itemize}

\section{Kluczowe funkcjonalności}

\subsection{Produkt}

\subsubsection{Edytowanie oferty sklepu}

Uprawiony do tego użytkownik powinien mieć możliwość dodawania nowych produktów, a także aktualizowania lub usuwania pozycji obecnych w ofercie.

\subsubsection{Przeglądanie listy produktów}

Klient powinien móc wyświetlać produkty z oferty, z możliwością wygodnego filtrowania ich np. według kategorii lub ceny. Dodatkowo powinien mieć możliwość posortowania ich względem wybranego porządku. 

\subsubsection{Wyświetlanie szczegółów produktu}

Po kliknięciu w pozycję z listy produktów, klient powinien ujrzeć jego szczegółowy opis. W szczególności jego cenę, informację o dostępności i listę opinii innych użytkowników.

\subsection{Koszyk}

W trakcie zakupów w e-sklepie klient powinien mieć możliwość swobodnego dodawania, usuwania i zmieniania liczebności produktów w koszyku. Widok koszyka powinien informować go o aktualnej łącznej kwocie zamówienia.

\subsection{Zamówienie}

\subsubsection{Utworzenie zamówienia}

Na podstawie koszyka klient powinien móc utworzyć zamówienie. W tym celu będzie musiał uzupełnić informacje o swoim adresie, wybranym sposobie dostawy i płatności, oraz zobowiązać się do uiszczenia zapłaty.

\subsubsection{Edytowanie stanu zamówienia}

Dla sprzedawców musi zostać udostępniona możliwość aktualizowania stanu realizacji zamówienia, który będzie widoczny dla użytkownika i zapewni mu komfort psychiczny.

\end{document}
\documentclass[12pt,a4paper]{article}
\usepackage{nopageno}
\usepackage[polish]{babel}
\usepackage[T1]{fontenc}
\usepackage[utf8]{inputenc}
\usepackage{amsmath,amsfonts}
\usepackage{titling}
\usepackage{mathtools}
\usepackage[margin=0.6in]{geometry} 
\usepackage{graphicx}
\usepackage{tikz}
\usepackage{nicefrac}


\title{AiSD Zadanie 6 L10}
\author{Cezary Świtała}

\begin{document}

\noindent
\textbf{Zadanie 6} Chcemy znaleźć krzywą regresji w postaci

\[
y = a + bx + cx^2,
\]
czyli zminimalizować wartość funkcji

\[
f(a,b,c) = \sum_{i=1}^{n} (a + bx_i + cx_i^2 - y_i)^2,
\]
a zatem rozwiązać układ równań
\[
\left\{
	\begin{array}{c}
		\nicefrac{\partial f}{\partial a} = 0 	\\
		\nicefrac{\partial f}{\partial b} = 0 	\\
		\nicefrac{\partial f}{\partial c} = 0 	\\
	\end{array}\right..
\]
Rozpiszmy pierwszą pochodną cząstkową
\begin{gather*}
	\frac{\partial f}{\partial a} = \sum_{i=1}^n 2( a + bx_i + c x_i^2 - y_i) \cdot 1 = 
	2 \left( \sum_{i=1}^n a + \sum_{i=1}^n bx_i + \sum_{i=1}^n c x_i^2 - \sum_{i=1}^n y_i \right) = \\
	= 2 \left( n \cdot a + b\sum_{i=1}^n x_i + c\sum_{i=1}^n x_i^2 - \sum_{i=1}^n y_i \right),
\end{gather*}
po przyrównaniu do zera
\[
	n \cdot a + b\sum_{i=1}^n x_1 + c\sum_{i=1}^n x_i^2 = \sum_{i=1}^n y_i.
\]
Podobnie rozpisujemy drugą pochodną

\begin{gather*}
	\frac{\partial f}{\partial b} = \sum_{i=1}^n 2( a + bx_i + c x_i^2 - y_i) \cdot x_i = 
	2 \left( \sum_{i=1}^n x_ia + \sum_{i=1}^n bx_i^2 + \sum_{i=1}^n c x_i^3 - \sum_{i=1}^n x_i y_i \right) = \\
	= 2 \left( a \sum_{i=1}^n x_i + b\sum_{i=1}^n x_i^2 + c\sum_{i=1}^n x_i^3 - \sum_{i=1}^n x_i y_i \right),
\end{gather*}
po przyrównaniu do zera
\[
	a \sum_{i=1}^n x_i + b\sum_{i=1}^n x_i^2 + c\sum_{i=1}^n x_i^3 = \sum_{i=1}^n x_i y_i.
\]
I jeszcze raz to samo z trzecią
\begin{gather*}
	\frac{\partial f}{\partial c} = \sum_{i=1}^n 2( a + bx_i + c x_i^2 - y_i) \cdot x_i^2 = 
	2 \left( \sum_{i=1}^n x_i^2 a + \sum_{i=1}^n bx_i^3 + \sum_{i=1}^n c x_i^4 - \sum_{i=1}^n y_i x_i^2 \right) = \\
	= 2 \left( a \sum_{i=1}^n x_i^2 + b\sum_{i=1}^n x_i^3 + c\sum_{i=1}^n x_i^4 - \sum_{i=1}^n y_i x_i^2 \right),
\end{gather*}
po przyrównaniu do zera
\[
	a \sum_{i=1}^n x_i^2 + b\sum_{i=1}^n x_i^3 + c\sum_{i=1}^n x_i^4 = \sum_{i=1}^n y_i x_i^2.
\]

\newpage
\noindent
Czyli pierwszemu układowi równań, równoważny jest 
 \[
\left\{
	\begin{array}{c}
		n \cdot a + b\sum x_i + c\sum x_i^2 = \sum y_i 	\\
		a \sum x_i + b\sum x_i^2 + c\sum x_i^3 = \sum y_i x_i 	\\
		a \sum x_i^2 + b\sum x_i^3 + c\sum x_i^4 = \sum y_i x_i^2	\\
	\end{array}\right..
\]
Można zapisać go w postaci macierzowej

\[
	\left[ \begin{array}{ccc}
		n\cdot a & b\sum x_i & c\sum x_i^2 \\
		a \sum x_i & b\sum x_i^2 & c\sum x_i^3 \\
		a \sum x_i^2 & b\sum x_i^3 & c\sum x_i^4 \\
	\end{array}\right] =
	\left[ \begin{array}{c}
		\sum y_i \\ 
		\sum x_i y_i \\
		\sum x_i^2 y_i
	\end{array}\right],
\]
następnie wyciągnąć wektor współczynników

\[
	\left[ \begin{array}{ccc}
		n & \sum x_i & \sum x_i^2 \\
		\sum x_i & \sum x_i^2 & \sum x_i^3 \\
		\sum x_i^2 & \sum x_i^3 & \sum x_i^4 \\
	\end{array}\right] \cdot 
	\left[ \begin{array}{c}
		a \\
		b \\
		c
	\end{array}\right]
	=
	\left[ \begin{array}{c}
		\sum y_i \\ 
		\sum x_i y_i \\
		\sum x_i^2 y_i
	\end{array}\right].
\]


\end{document}
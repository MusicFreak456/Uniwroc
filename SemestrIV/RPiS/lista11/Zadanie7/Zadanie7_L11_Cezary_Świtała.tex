\documentclass[12pt,a4paper]{article}
\usepackage{nopageno}
\usepackage[polish]{babel}
\usepackage[T1]{fontenc}
\usepackage[utf8]{inputenc}
\usepackage{amsmath,amsfonts}
\usepackage{titling}
\usepackage{mathtools}
\usepackage[margin=0.6in]{geometry} 
\usepackage{graphicx}
\usepackage{tikz}
\usepackage{nicefrac}


\title{RPIS Zadanie 7 L11}
\author{Cezary Świtała}

\begin{document}

\noindent
\textbf{Zadanie 7.11} Mamy sprawdzić, że \( U,V \sim N(0,1) \), gdzie

\[
	U = \frac{1}{2 \sqrt{15}} (-3Y_1 + 2Y_2),\ V = \frac{1}{2\sqrt{20}} (3Y_1 + 2Y_2 - 12).
\]

\noindent
Wiemy, że \( Y = \left[ \begin{array}{cc}
	Y_1 \\
	Y_2
\end{array} \right] \) ma rozkład \( N(\mu, \Sigma) \), gdzie \( \mu = \left[ \begin{array}{cc}
	2 \\
	3
\end{array} \right] \) oraz \( \Sigma = \left[ \begin{array}{cc}
	4 & 1 \\
	1 & 9
\end{array} \right] \).
Niech \(X = \left[ \begin{array}{c}
	U \\
	V
\end{array} \right]  = AY + c \), gdzie 

\[
	A = \left[ \begin{array}{cc}
			\frac{-3}{2\sqrt{15}} & \frac{1}{\sqrt{15}} \\
			\frac{3}{2\sqrt{21}}  & \frac{1}{\sqrt{21}} 
		\end{array}	 \right],
\]
a wektor $c$ to
\[
	c = \left[ \begin{array}{c}
			0 \\
			\frac{-6}{\sqrt{21}}
		\end{array}	 \right].
\]
Niech \( Z = AY = \left[ \begin{array}{c}
	U' \\
	V'
\end{array} \right] \), jeśli \(A\) jest odwracalna, to na podstawie twierdzenia z wykładu
\[ 
	Z \sim N(A\mu, A \Sigma A^T) 
\] 

\noindent
\( |A| \approx -0.17 \), czyli macierz jest odwracalna. Zatem policzmy macierz wartości oczekiwanych zmiennej \( Z \)

\[
	A\mu = \left[ \begin{array}{cc}
			\frac{-3}{2\sqrt{15}} & \frac{1}{\sqrt{15}} \\
			\frac{3}{2\sqrt{21}}  & \frac{1}{\sqrt{21}} 
		\end{array}	 \right] \cdot 
		\left[ \begin{array}{cc}
			2 \\
			3
		\end{array} \right]	=
		\left[ \begin{array}{c}
			0 \\
			\frac{6}{\sqrt{21}}
		\end{array}	\right].
\]
Następnie macierz kowariancji
\begin{gather*}
	A \Sigma A^T = \left[ \begin{array}{cc}
			\frac{-3}{2\sqrt{15}} & \frac{1}{\sqrt{15}} \\
			\frac{3}{2\sqrt{21}}  & \frac{1}{\sqrt{21}} 
		\end{array}	 \right] \cdot \left[ \begin{array}{cc}
			4 & 1 \\
			1 & 9
		\end{array} \right] \cdot \left[ \begin{array}{cc}
			\frac{-3}{2\sqrt{15}} & \frac{3}{2\sqrt{21}} \\
			\frac{1}{\sqrt{15}}  & \frac{1}{\sqrt{21}} 
		\end{array}	 \right] = \\ =
		\left[ \begin{array}{cc}
			\frac{-5}{\sqrt{15}} & \frac{15}{\sqrt{15}} \\
			\frac{7}{\sqrt{21}}  & \frac{21}{2\sqrt{21}} 
		\end{array}	 \right] \cdot 
		\left[ \begin{array}{cc}
			\frac{-3}{2\sqrt{15}} & \frac{3}{2\sqrt{21}} \\
			\frac{1}{\sqrt{15}}  & \frac{1}{\sqrt{21}} 
		\end{array}	 \right] = \left[ \begin{array}{cc}
			1 & 0 \\
			0 & 1 		
		\end{array}		 \right].
\end{gather*}

\noindent
Korzystając z powyższych równań jesteśmy już w stanie odczytać rozkład zmiennej losowej \( U' = U \sim N(0,1) \) oraz \( V' \sim N \left( \frac{6}{\sqrt{21}}, 1 \right)  \). 

Żeby wyznaczyć rozkład zmiennej $V$ możemy skorzystać z faktu, że jeśli zmienna losowa $T$ należy do rozkładu \( N(\mu, \sigma^2) \), to zmienna \(T + b\) będzie należeć do rozkładu \( N(\mu + b , \sigma^2) \).

\noindent
\textbf{Dowód}. Łatwy dowód można przeprowadzić na MGF'ach

\begin{gather*}
	M_{T + b}(t) = e^{tb} \cdot M_T(t) = exp(tb) \cdot exp\left( t\mu + \frac{\sigma^2 t^2}{2} \right) =\\=
	exp\left( t(\mu + b) + \frac{\sigma^2 t^2}{2} \right)
\end{gather*}
Wynik jest MGF'em rozkładu \( N( \mu + b, \sigma^2) \), co kończy dowód. 

Korzystając z powyższego
\[
	V = V' + \frac{-6}{\sqrt{21}}  \sim N \left( \frac{6}{\sqrt{21}} + \frac{-6}{\sqrt{21}}, 1 \right) = N(0,1).
\]



\end{document}
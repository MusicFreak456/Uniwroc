\documentclass[12pt,a4paper]{article}
\usepackage[polish]{babel}
\usepackage[T1]{fontenc}
\usepackage[utf8]{inputenc}
\usepackage{amsmath,amsfonts}
\usepackage{titling}
\usepackage{textcomp}
\usepackage{gensymb}
\usepackage{mathtools}
\usepackage[margin=0.5in]{geometry} 

\pagestyle{empty}

\begin{document}
\noindent
\textbf{Zadanie 7.}
Sprawdź, czy podanie poniżej macierze są dodatnio określone:
\[
\left[
\begin{array}{ccc}
1  & 2 & 0\\
-2 & 2 & 0\\
0  & 0 & 0
\end{array}\right]
,
\left[
\begin{array}{ccc}
2 & 2 & 0\\
2 & 2 & 0\\
0 & 0 & 1
\end{array}\right]
,
\left[
\begin{array}{ccc}
6 & 2 & 4\\
2 & 1 & 1\\
4 & 1 & 5
\end{array}\right]
,
\left[
\begin{array}{cccc}
6 & 7  & 3 & 3\\
7 & 15 & 7 & 3\\
3 & 7  & 11 & 1\\
3 & 3  & 1  & 2
\end{array}\right]
.
\]
\textbf{Rozwiązanie.}

\textbf{Pierwsza macierz:}
Z \textbf{Faktu 12.6} wiemy, że macierz jest dodatnio określona wtedy i tylko wtedy gdy jest symetryczna. Nasza macierz nie jest symetryczna, bo nie jest równa swojej transpozycji, zatem nie jest dodatnio określona.

\textbf{Druga macierz:} W tej, tak samo jak przy dwóch następnych macierzach, skorzystamy z kryterium Sylvestera, które mówi, że macierz symetryczna jest dodatnio określona wtedy i tylko wtedy gdy wyznaczniki jej wiodących minorów głównych są dodatnie.
Badamy zatem kolejne wyznaczniki:
\[
M_1=
\left[
\begin{array}{c}
2
\end{array}\right]
\]
\[
det(M_1)=2
\]
\[
M_2=
\left[
\begin{array}{cc}
2 & 2 \\
2 & 2 \\
\end{array}\right]
\]
\[
det(M_2) = 2 \cdot 2 - 2 \cdot 2 = 0 
\]
Wyznacznik drugiego minora nie jest większy od zera, więc na mocy kryterium Sylvestera macierz nie jest dodatnio określona.

\textbf{Trzecia macierz:}
\[
M_1=
\left[
\begin{array}{c}
6
\end{array}\right]
\]
\[
det(M_1)=6
\]
\[
M_2=
\left[
\begin{array}{cc}
6 & 2 \\
2 & 1 \\
\end{array}\right]
\]
\[
det(M_2) = 6 \cdot 1 - 2 \cdot 2 = 2 
\]
\[
M_3=
\left[
\begin{array}{ccc}
6 & 2 & 4\\
2 & 1 & 1\\
4 & 1 & 5
\end{array}\right]
\]
\[
det(M_3)= 30 + 8 + 8 - 16 - 20 - 6 = 4  
\]
Wszystkie wyznaczniki powyższych minorów są dodatnie, więc macierz jest dodatnio określona.

\textbf{Czwarta macierz:}
\[
M_1=
\left[
\begin{array}{c}
6
\end{array}\right]
\]
\[
det(M_1)=6
\]
\[
M_2=
\left[
\begin{array}{cc}
6 & 7 \\
7 & 15 \\
\end{array}\right]
\]
\[
det(M_2) = 6 \cdot 15 - 7 \cdot 7 = 90 - 49 = 41 
\]
\[
M_3=
\left[
\begin{array}{ccc}
6 & 7 & 3\\
7 & 15 & 7\\
3 & 7 & 11
\end{array}\right]
\]
\[
det(M_3)= 990 + 147 + 147 - 135 - 294 - 539 = 316
\]
\newpage
\[
M_4=
\left[
\begin{array}{cccc}
6 & 7  & 3 & 3\\
7 & 15 & 7 & 3\\
3 & 7  & 11 & 1\\
3 & 3  & 1  & 2
\end{array}\right]
\]
Do policzenia wyznacznika skorzystamy z rozwinięcia Laplace'a, ale najpierw przekształcimy macierz do wygodniejszej postaci.
\[
\left[
\begin{array}{cccc}
6 & 7  & 3 & 3\\
7 & 15 & 7 & 3\\
3 & 7  & 11 & 1\\
3 & 3  & 1  & 2
\end{array}\right]
\xrightarrow{(1) = (1) - 2(4)}
\left[
\begin{array}{cccc}
0 & 1  & 1 & -1\\
7 & 15 & 7 & 3\\
3 & 7  & 11 & 1\\
3 & 3  & 1  & 2
\end{array}\right]
\xrightarrow{(3) = (3) - (4)}
\left[
\begin{array}{cccc}
0 & 1  & 1 & -1\\
7 & 15 & 7 & 3\\
0 & 4  & 10 & -1\\
3 & 3  & 1  & 2
\end{array}\right]
\]
Teraz rozwijamy względem pierwszej kolumny.
\[
det(M_4)= (-1)^{2+1} \cdot 7 \cdot
\left|
\begin{array}{ccc}
1 & 1 & -1\\
4 & 10 & -1\\
3 & 1 & 2
\end{array}\right|
+(-1)^{1+4}\cdot 3\cdot
\left|
\begin{array}{ccc}
1 & 1 & -1\\
15 & 7 & 3\\
4 & 10 & -1
\end{array}\right|
=
\]
\[
= -7\cdot (20 - 3 - 4 + 30 -8 +1) - 3 \cdot (-7 + 12 - 150 +28 + 15 - 30) =
\]
\[
= - 7 \cdot 36 - 3 \cdot (-132) = 144
\]

Znów wszystkie wyznaczniki powyższych minorów są dodatnie, więc macierz jest dodatnio określona.

\end{document}

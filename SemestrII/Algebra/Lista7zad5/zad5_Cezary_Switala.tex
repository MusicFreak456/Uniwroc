\documentclass[12pt,a4paper]{article}
\usepackage[polish]{babel}
\usepackage[T1]{fontenc}
\usepackage[utf8]{inputenc}
\usepackage{amsmath,amsfonts}
\usepackage{titling}
\usepackage{textcomp}
\usepackage{gensymb}
\usepackage{mathtools}
\usepackage[margin=0.5in]{geometry} 

\pagestyle{empty}

\begin{document}
\noindent
\textbf{Zadanie 5. }
Znajdź wartości własne, ich krotności algebraiczne i geometryczne dla poniższych macierzy:
\[
\left[
\begin{array}{ccc}
7 & -12 & 6\\
10 & -19 & 10\\
12 & -24 & 13
\end{array}\right]
,
\left[
\begin{array}{ccc}
2 & -1 & 2\\
5 & -3 & 3\\
-1 & 0 & -2
\end{array}\right]
,
\left[
\begin{array}{ccc}
0 & 1 & 0\\
-4 & 4 & 0\\
-2 & 1 & 2
\end{array}\right]
.
\]
Dla jednej z wartości oblicz odpowiadające wektory własne.

\vskip 0.4cm
\noindent
\textbf{Rozwiązanie.}
Z \textbf{Lematu 8.10} wiemy, że da macierzy kwadratowej \(\lambda\) jest wartością własną wtedy i tylko wtedy, gdy jest pierwiastkiem wielomianu charakterystycznego tej macierzy, więc zaczniemy od jego wyznaczenia metodą Sarrusa.
\[
A=
\left[
\begin{array}{ccc}
7 & -12 & 6\\
10 & -19 & 10\\
12 & -24 & 13
\end{array}\right]
~~A-\lambda Id=
\left[
\begin{array}{ccc}
7 - \lambda & -12 & 6\\
10 & -19 - \lambda & 10\\
12 & -24 & 13 - \lambda
\end{array}\right]
\]
\[
\begin{array}{c}
det(A-\lambda Id) = (7-\lambda) (-19-\lambda)(13 - \lambda) - 1440 - 1440 -72(-19 - \lambda)+240(7-\lambda)+120(13 - \lambda)= \\
= (-133 + 12\lambda + \lambda^2) (13 - \lambda) - 2880 + 1368 + 72\lambda + 1680 - 240\lambda + 1560 - 120\lambda=\\
= (\lambda^2 + 12\lambda -133) (13 - \lambda) - 288\lambda + 1728=\\
= 13\lambda^2 - \lambda^3 + 156\lambda - 12\lambda^2 - 1729 + 133\lambda - 288\lambda + 1728 = \\
= -\lambda^3 + \lambda^2 + \lambda -1 = \\
= -\lambda^2(\lambda-1) + \lambda - 1 = \\
= (\lambda - 1) (1-\lambda^2)
\end{array}
\]
Zauważamy, że wielomian zeruje się dla \(\lambda\) równej 1, lub -1, są to zatem wartości własne macierzy. Z wielomianu możemy również odczytać ich krotności algebraiczne. Dla -1 wynosi ona 1, a więc korzystając z uwagi przy \textbf{Lemacie 8.17} możemy powiedzieć, że jej krotność geometryczna również wynosi 1. Dla 1 krotność algebraiczna wynosi 2, a jej krotność geometryczną ustalimy przy okazji wyznaczania wektorów własnych.

Zbiór wektorów własnych to \(ker(A-\lambda Id)\) (\textbf{Lemat 8.14}), a więc by go wyznaczyć rozwiążemy równanie
\[
(A-Id)\vec{X}=\vec{0}
\]
W tym celu doprowadzimy macierz do postaci schodkowej
\[
\left[
\begin{array}{ccc}
6 & -12 & 6\\
10 & -20 & 10\\
12 & -24 & 12
\end{array}\right]
\xrightarrow{(3) = (3) - 2(1)}
\left[
\begin{array}{ccc}
6 & -12 & 6\\
10 & -20 & 10\\
0 & 0 & 0
\end{array}\right]
\xrightarrow{(2) = \frac{1}{2} (2)}
\left[
\begin{array}{ccc}
6 & -12 & 6\\
5 & -10 & 5\\
0 & 0 & 0
\end{array}\right]
\xrightarrow{(1)=(1)-(2)}
\left[
\begin{array}{ccc}
1 & -2 & 1\\
5 & -10 & 5\\
0 & 0 & 0
\end{array}\right]
\]
\[
\xrightarrow{(2)=(2)-5(1)}
\left[
\begin{array}{ccc}
1 & -2 & 1\\
0 & 0 & 0\\
0 & 0 & 0
\end{array}\right]
\]
Macierz pomnożoną przez wektor zmiennych przyrównujemy do wektora zerowego zatem otrzymujemy równanie
\[
\
x_1 - 2x_2 + x_3 = 0\\
\]
Z którego mamy \(x_1 = 2x_2-x_3\), więc zbiór wektorów własnych ma postać
\[
\{(2x_2-x_3,x_2,x_3)| x_2,x_3\in \mathbb{R}\}=
\{(2x_2,x_2,0)+(-x_3,0,x_3)| x_2,x_3\in \mathbb{R}\}
\]
Widać, że jego baza to \(\{(2,1,0),(-1,0,1)\}\), co na podstawie \textbf{Faktu 8.16} implikuje, że krotność geometryczna 1 wynosi 2.
\newpage
Kolejne dwie macierze rozwiązuje się zupełnie tak samo. Przedstawię więc same rachunki, bez podobnych do powyższych komentarzy.
\[
B=
\left[
\begin{array}{ccc}
2 & -1 & 2\\
5 & -3 & 3\\
-1 & 0 & -2
\end{array}\right]
~~B-\lambda Id=
\left[
\begin{array}{ccc}
2-\lambda & -1 & 2\\
5 & -3-\lambda & 3\\
-1 & 0 & -2-\lambda
\end{array}\right]
\]
\[
\begin{array}{c}
det(B-\lambda Id) = (2-\lambda)(-3-\lambda)(-2-\lambda) + 3 + 2(-2-\lambda) + 5(-2-\lambda)=\\
=(-6-2\lambda+3\lambda+\lambda^2)(-2-\lambda)+3-6-2\lambda-10-5\lambda=\\
=-2\lambda^2-\lambda^3-2\lambda-\lambda^2+12+6\lambda-13-7\lambda=\\
=-\lambda^3-3\lambda^2-3\lambda-1=\\
=-(\lambda+1)^3
\end{array}
\]
Jedna wartość własna równa -1 o krotności algebraicznej 3. Analogicznie do poprzedniego przykładu wyznaczamy zbiór wektorów własnych.

\[
(B+Id)\vec{X}=\vec{0}
\]

\[
B+Id=
\left[
\begin{array}{ccc}
3 & -1 & 2\\
5 & -2 & 3\\
-1 & 0 & -1
\end{array}\right]
\xrightarrow{(1)=(1)+3(3), (2)=(2)+5(3)}
\left[
\begin{array}{ccc}
0 & -1 & -1\\
0 & -2 & -2\\
-1 & 0 & -1
\end{array}\right]
\xrightarrow{(2)=(2)-2(1)}
\left[
\begin{array}{ccc}
0 & -1 & -1\\
0 & 0 & 0\\
-1 & 0 & -1
\end{array}\right]
\]
\[
\left\{ 
\begin{array}{llllllll}
-x_2 - x_3 = 0 & \rightarrow & x_2=-x_3\\
-x_1 - x_3 = 0 & \rightarrow & x_1=-x_3
\end{array} \right.
\]
Zbiór wektorów własnych (\(ker(B+Id)\)):
\[
\{(-x_3,-x_3,x_3)| x_3\in \mathbb{R}\}
\]
Baza to zbiór \(\{(-1,-1,1)\}\) o mocy 1, zatem krotność geometryczna wartości własnej -1 to 1. Kolejny przykład analogicznie:
\[
C=
\left[
\begin{array}{ccc}
0 & 1 & 0\\
-4 & 4 & 0\\
-2 & 1 & 2
\end{array}\right]
~~C-\lambda Id=
\left[
\begin{array}{ccc}
-\lambda & 1 & 0\\
-4 & 4-\lambda & 0\\
-2 & 1 & 2-\lambda
\end{array}\right]
\]

\[
\begin{array}{c}
det(C-\lambda Id) = -\lambda(4-\lambda)(2-\lambda)+4(2-\lambda)=\\
=-\lambda(8-4\lambda-2\lambda+\lambda^2)+8-4\lambda=\\
=-\lambda^3+6\lambda^2-8\lambda+8-4\lambda=\\
=-\lambda^3+6\lambda^2-12\lambda+8=\\
=-(\lambda-2)^3
\end{array}
\]
Mamy wartość własną 2 o krotności algebraicznej 3.
\[
(C-2Id)\vec{X}=\vec{0}
\]

\[
C-2Id=
\left[
\begin{array}{ccc}
-2 & 1 & 0\\
-4 & 2 & 0\\
-2 & 1 & 0
\end{array}\right]
\xrightarrow{(2)=(2)-2(1),(3)=(3)-(1)}
\left[
\begin{array}{ccc}
-2 & 1 & 0\\
0 & 0 & 0\\
0 & 0 & 0
\end{array}\right]
\]
\[
-2x_1+x_2 = 0 \rightarrow x_1=\frac{1}{2}x_2
\]

Zbiór wektorów własnych (\(ker(C-2Id)\)):
\[
\{(\frac{1}{2}x_2,x_2,x_3)| x_3,x_2\in \mathbb{R}\} = 
\{(\frac{1}{2}x_2,x_2,0) + (0,0,x_3)| x_3,x_2\in \mathbb{R}\}=
\]
\[
\{x_2( \frac{1}{2},1,0) + x_3(0,0,1)| x_3,x_2\in \mathbb{R}\}
\]
Baza to zbiór \(\{(\frac{1}{2},1,0),(0,0,1)\}\) a jej moc to 2, zatem krotność geometryczna wartości własnej 2 to 2.
\end{document}
\documentclass[12pt,a4paper]{article}
\usepackage[polish]{babel}
\usepackage[T1]{fontenc}
\usepackage[utf8]{inputenc}
\usepackage{amsmath,amsfonts}
\usepackage{titling}
\usepackage{textcomp}
\usepackage{gensymb}
\usepackage{mathtools}
\usepackage[margin=0.5in]{geometry} 

\pagestyle{empty}




\begin{document}
\noindent
\textbf {Zadanie 5. }
Ile rozwiązań ma poniższy układ w zależności od parametru \(\lambda\)? Układ jest nad \(\mathbb{Z}_{13}\), tym samym \(\lambda\in\mathbb{Z}_{13}\).

\[
\left\{ 
\begin{array}{llllllll}
\lambda x & + & \lambda^2y & + &\lambda^3z & = & 1\\ 
x & + & \lambda^2y & + &\lambda^3z & = & \lambda\\ 
x & + & y& + &\lambda^3z & = & \lambda^2
\end{array} \right.
\]

\textbf{Rozwiązanie. }
Zacznijmy od przedstawienie powyższego układu w postaci macierzowej \(A\vec{X}=\vec{B}\).

\[
A=
\left[
\begin{array}{ccc}
\lambda & \lambda^2 & \lambda^3\\
1 & \lambda^2 & \lambda^3 \\
1 & 1 & \lambda^3
\end{array}\right]
,\vec{X}=
\left[
\begin{array}{c}
x\\
y\\
z\\
\end{array}\right]
,\vec{B}=
\left[
\begin{array}{c}
1\\
\lambda\\
\lambda^2
\end{array}\right]
\]
Z \textbf{Twierdzenia 7.1} wiemy, że jeśli \(det(A)\ne0\) to równanie ma jedno rozwiązanie. Wyznaczmy zatem \(det(A)\) korzystając z reguły Sarrusa.

\[
\begin{array}{c}
det(A)=\lambda^6 + \lambda^5 + \lambda^3 - \lambda^5 - \lambda^4-\lambda^5 =\\
=\lambda^6 - \lambda^5 -\lambda^4 +\lambda^3=\\
=\lambda^3( \lambda^3 - \lambda^2 - \lambda + 1)=\\
=\lambda^3(\lambda^2(\lambda - 1) - (\lambda - 1))=\\
=\lambda^3(\lambda^2-1)(\lambda-1)
\end{array}
\]
Zauważamy, że nasz wyznacznik zeruje się dla \(\lambda = 0\), \(\lambda = 1\) lub \(\lambda=-1\), zatem dla innych wartości parametru \(\lambda\) jest różny od zera, czyli istnieje dokładnie jedno rozwiązanie. Pozostaje nam zbadać ile rozwiązań pojawi się w pozostałych przypadkach. W tym celu dla każdego przypadku policzymy \(rk(A)\) oraz \(rk(A|B)\) stosując eliminację Gaussa na wierszach.
\vskip 0.4cm
\noindent
\textbf{\(1\degree\)} \(\lambda=-1\)
\[
A=
\left[
\begin{array}{ccc}
-1 & 1 & -1\\
1 & 1 & -1 \\
1 & 1 & -1
\end{array}\right]
\xrightarrow{(2)=(2)-(3)}
\left[
\begin{array}{ccc}
-1 & 1 & -1\\
0 & 0 & 0 \\
1 & 1 & -1
\end{array}\right]
\xrightarrow{(1)=(1)+(3)}
\left[
\begin{array}{ccc}
0 & 2 & -2\\
0 & 0 & 0 \\
1 & 1 & -1
\end{array}\right]
\xrightarrow{}
\left[
\begin{array}{ccc}
1 & 1 & -1\\
0 & 2 & -2\\
0 & 0 & 0 
\end{array}\right]
\]
Jak widać, w tym przypadku \(rk(A)=2\). Te same operacje zastosujemy na macierzy rozszerzonej.
\[
\left[
\begin{array}{cccc}
-1 & 1 & -1 & 1\\
1 & 1 & -1 & -1\\
1 & 1 & -1 & 1
\end{array}\right]
\xrightarrow{}
\left[
\begin{array}{cccc}
-1 & 1 & -1 &1\\
0 & 0 & 0 &-2\\
1 & 1 & -1 & 1
\end{array}\right]
\xrightarrow{}
\left[
\begin{array}{cccc}
0 & 2 & -2 & 2\\
0 & 0 & 0 & -2\\
1 & 1 & -1 & 1
\end{array}\right]
\xrightarrow{}
\left[
\begin{array}{cccc}
1 & 1 & -1 & 1\\
0 & 2 & -2 & 2\\
0 & 0 & 0 &-2 
\end{array}\right]
\]
Okazuje się, że macierz \(A|B\) ma rząd równy 3, a więc \(rk(A)\ne rk(A|B)\), czyli na podstawie\\ \textbf{Faktu 7.4} mówimy, że układ nie ma rozwiązania dla \(\lambda=-1\).
\vskip 0.4cm
\noindent
\textbf{\(2\degree\)} \(\lambda=0\)
\[
A=
\left[
\begin{array}{ccc}
0 & 0 & 0\\
1 & 0 & 0\\
1 & 1 & 0
\end{array}\right]
\xrightarrow{(3)=(3)-(2)}
\left[
\begin{array}{ccc}
0 & 0 & 0\\
1 & 0 & 0\\
0 & 1 & 0
\end{array}\right]
\xrightarrow{}
\left[
\begin{array}{ccc}
1 & 0 & 0\\
0 & 1 & 0\\
0 & 0 & 0
\end{array}\right]
\]
\(rk(A)=2\), podobnie jak wcześniej sprawdzamy \(A|B\).
\[
A=
\left[
\begin{array}{cccc}
0 & 0 & 0 & 1\\
1 & 0 & 0 & 0\\
1 & 1 & 0 & 0
\end{array}\right]
\xrightarrow{(3)=(3)-(2)}
\left[
\begin{array}{cccc}
0 & 0 & 0 & 1\\
1 & 0 & 0 & 0\\
0 & 1 & 0 & 0
\end{array}\right]
\xrightarrow{}
\left[
\begin{array}{cccc}
1 & 0 & 0 & 0\\
0 & 1 & 0 & 0\\
0 & 0 & 0 & 1
\end{array}\right]
\]
I znów \(rk(A|B) = 3\), więc \(rk(A)\ne rk(A|B)\) i układ nie ma rozwiązania dla \(\lambda = 0\).

\newpage
\noindent
\textbf{\(3\degree\)} \(\lambda=1\)
\[
A=
\left[
\begin{array}{ccc}
1 & 1 & 1\\
1 & 1 & 1\\
1 & 1 & 1
\end{array}\right]
\xrightarrow{(2)=(2)-(1),(3)=(3)-(1)}
\left[
\begin{array}{ccc}
1 & 1 & 1\\
0 & 0 & 0\\
0 & 0 & 0
\end{array}\right]
\]
Rząd \(A\) to tym razem 1, sprawdzamy rząd \(A|B\).
\[
A=
\left[
\begin{array}{cccc}
1 & 1 & 1 & 1\\
1 & 1 & 1 & 1\\
1 & 1 & 1 & 1
\end{array}\right]
\xrightarrow{(2)=(2)-(1),(3)=(3)-(1)}
\left[
\begin{array}{cccc}
1 & 1 & 1 & 1\\
0 & 0 & 0 & 0\\
0 & 0 & 0 & 0
\end{array}\right]
\]
Mamy \(rk(A) = rk(A|B)\) zatem istnieje co najmniej jedno rozwiązanie. Ustalimy zatem zbiór rozwiązań.\\
Po eliminacji mamy równanie \(x+y+z=1\), możemy wyznaczyć z niego \(x\). Otrzymujemy \(x=1-y-z\).
Zatem zbiór rozwiązań będzie miał postać:
\[
\{(1-y-z,y,z) | y,z\in \mathbb{Z}_{13}\}
\]
Zmienne \(y\) i \(x\) możemy ustalić na \(13^2\) możliwości, więc dla \(\lambda=1\) mamy 169 możliwych rozwiązań.



\end{document}
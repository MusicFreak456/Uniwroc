\documentclass[12pt,a4paper]{article}
\usepackage[polish]{babel}
\usepackage[T1]{fontenc}
\usepackage[utf8]{inputenc}
\usepackage{amsmath,amsfonts}
\usepackage{titling}
\usepackage{textcomp}
\usepackage{gensymb}
\usepackage{mathtools}
\usepackage[margin=0.43in]{geometry} 
\usepackage{ulem}

\pagestyle{empty}

\begin{document}


\noindent
\textbf{Zadanie 10.} Centralizatorem elementu \(a\) w grupie \(G\) nazywamy zbiór elementów przemiennych z \(a\), czyli
\[
	G(a) = \{ b \in G : ab = ba\}
\]
Centrum grupy \(G\) nazywamy zbiór
\[
	Z(G) = \{a : \forall b \in G : ab = ba\}
\]
(czyli: przemiennych ze wszystkimi elementami w \(G\)). Udowodnij, że dla dowolnej grupy \(G\) i elementu \(a\) centralizator \(G(a)\) oraz centrum \(Z(G)\) są podgrupami \(G\). Pokaż też, że 
\[
Z(G) = \bigcap_{g\in G} G(g) .
\]

\noindent
\textbf{Rozwiązanie: Część pierwsza} 

W pierwszej części zadania pokażemy, że \(G(a)\) oraz \(Z(G)\) są podgrupami \(G\) dla dowolnej grupy \(G\) oraz elementu \(a\). Zacznijmy od centralizatora. Weźmy dowolną grupę G i element \(a \in G\). Aby zbiór \(G(a)\) był podgrupą muszą zostać spełnione poniższe warunki:

\vskip 0.4cm
\textbf{1. G(a) zawiera element neutralny.} Wiemy, że dla elementu neutralnego \(e \in G\) prawdziwa jest równość \( ae = ea\). Zatem na pewno \(e \in G(a)\).

\vskip 0.4cm
\textbf{2. G(a) zamknięty na działanie.} Weźmy dowolne elementy \(x,y \in G(a)\), pokażemy że \(xy \in G(a)\), czyli że \(a(xy) = (xy)a\). Korzystając z łączności działania i definicji centralizatora:
\[
a(xy) = (ax)y = (xa)y = x(ay) = x(ya) = (xy)a
\]
Zatem \(xy \in G(a)\).

\vskip 0.4cm
\textbf{3. Każdy element w G(a) ma element  \sout{przeciwny} odwrotny} Weźmy dowolny element \(x \in G(a)\), pokażemy że \(x^{-1} \in G(a)\). Z definicji centralizatora:
\[
\begin{array}{cc}
ax = xa & \cdot /x^{-1} \\
x^{-1}ax = x^{-1} x a & \\
x^{-1}ax = ea & \\
x^{-1}ax = a & \cdot /x^{-1} \\
x^{-1}axx^{-1} = ax^{-1} & \\
x^{-1}ae = ax^{-1} & \\
x^{-1}a = ax^{-1} & \\
\end{array}
\]
Zatem \(x^{-1} \in G(a) \).


Ze spełnienia powyższych warunków wnioskujemy, że 
\[
G(a) \leq G.
\]

Dla centrum dowód będzie wyglądał bardzo podobnie. Weźmy dowolną grupę G i sprawdźmy czy zachodzą poniższe warunki:

\vskip 0.4cm
\textbf{1. Z(G) zawiera element neutralny.} Wiemy, że dla dowolnego elementu \(x \in G \) element neutralny \(e \in G \) spełnia równość \( ex = xe \), zatem \( e \in Z(G) \).

\vskip 0.4cm
\textbf{2. Z(G) zamknięty na działanie.} Weźmy dowolne elementy \(x,y \in Z(G)\), pokażemy że \(xy \in G(a)\), czyli że dla dowolnego \(a \in G\) \((xy)a = a(xy)\). Korzystając z łączności działania i definicji centrum mamy:
\[
(xy)a = x(ya) = x(ay) = (xa)y = (ax)y = a(xy)
\]
Zatem \(xy \in Z(G)\).

\vskip 0.4cm
\textbf{3. Każdy element w Z(G) ma element \sout{przeciwny} odwrotny} Weźmy dowolny element \(x \in Z(G)\), pokażemy że \(x^{-1} \in Z(G)\). Z definicji centrum \(\forall a \in G\) mamy:
\[
\begin{array}{cc}
xa = ax & \cdot /x^{-1} \\
x^{-1}xa = x^{-1} a x & \\
ea = x^{-1} a x & \\
a = x^{-1} a x & \cdot /x^{-1} \\
ax^{-1} = x^{-1} a x x^{-1} & \\
ax^{-1} = x^{-1}ae & \\
ax^{-1} = x^{-1}a & \\
\end{array}
\]
Zatem \(x^{-1} \in Z(G) \).

Tak jak poprzednio wnioskujemy z tego, że
\[
Z(G) \leq G.
\]

\noindent
\textbf{Rozwiązanie: Część druga}


W drugiej części pokazać mamy, że dla dowolnej grupy \(G\) zachodzi równość
\[
Z(G) = \bigcap_{g\in G} G(g) .
\]
Jest to równość zbiorów, zatem wystarczy że zachodzić będzie zawieranie w obie strony. 

\vskip 0.4cm
Pokażemy więc najpierw, że \(Z(G) \subseteq \bigcap_{g\in G} G(g)\), czyli że \( \forall_{x \in G} \): \( x \in Z(G) \implies \bigcap_{g\in G} G(g)\).
Niech \(G\) będzie dowolną grupą, a \(x\) będzie dowolnym elementem z \(Z(G)\):
\[
x \in Z(G) \iff \forall_{g \in G}: \ xg = gx \iff \forall_{g \in G}: \ x \in G(g)
\iff x \in \bigcap_{g \in G} G(g)
\]
Zatem \(Z(G) \subseteq \bigcap_{g\in G} G(g)\). Można również zauważyć, że wszystkie zastosowane przez nas przejścia działają w obie strony, zatem pokazują one jednocześnie że \(x \in \bigcap_{g \in G} G(g) \implies x \in Z(G)\), czyli że \(Z(G) \supseteq \bigcap_{g\in G} G(g)\). Z faktu zawierania w obie strony wnioskujemy, że zachodzi równość tych dwóch, co kończy dowód.

No i siup

\end{document}